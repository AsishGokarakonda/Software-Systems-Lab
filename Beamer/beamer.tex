\documentclass{beamer}
\usepackage{url}
\usepackage{colortbl}
\usepackage{xcolor}
\usepackage{amsmath}
\usepackage{hyperref}
\usepackage{lipsum}
\usepackage{mathtools}
\usetheme{Warsaw}
\usefonttheme{serif}

%from here title page begins
%below commands are used to display title,subtitle,author,institute,email id and date in order
\title{Assignment-2} 
\subtitle{Software Systems Lab}
\author{\textit{"Gokarakonda Sri Sai Asish"}}
\institute{IIT Dharwad \newline \url{https://www.iitdh.ac.in/}}
\date{August 11, 2021}

\begin{document}
\begin{frame}%this is to create frame
\titlepage 
\begin{figure}
\centering%this is for centering the figure
  \includegraphics[height=2.5cm,width=2.9cm]{Image 2.jpeg} %this is to insert a figure
\end{figure}
\end{frame}


\begin{frame}{Dynamic Programming}%this is to create frame
\transblindshorizontal%transistions
\label{dyano}
\begin{itemize}%this is to create list of items
\item Characteristics of Dynamic Programming
   \begin{enumerate}
    \item \textit{Overlapping Sub-problems} 
      \begin{block}{1}%this is to create a block
      Subproblems are smaller versions of the original problem. Any problem has overlapping sub-problems if finding its solution involves solving the same subproblem multiple times. 
      \end{block}
      \item \textit{Optimal Substructure}      
      \begin{block}{2}%this is to create a block
        Any problem has optimal substructure property if its overall optimal solution can be constructed from the optimal solutions of its subproblems.
      \end{block}
   \end{enumerate}
   \end{itemize}
\end{frame}


\begin{frame}[t]{DP Methods}%this is to create frame
\transblindshorizontal%transistions
\vspace{10pt}
\begin{itemize}%this is to create list of items
  \item \textbf{Top-down with Memoization}  %this is used to make text bold
   \begin{block}{1}%this is to create a block
   In this approach, we try to solve the bigger problem by recursively finding the solution to smaller sub-problems. Whenever we solve a sub-problem, we cache its result so that we don't end up solving it repeatedly if it's called multiple times. Instead, we can just return the saved result.
   \end{block}
   \only<2>{\item \textbf{Bottom-up with Tabulation}%this is used to make text bold
   \begin{alertblock}{2}%this is to create block with red color
   Tabulation is the opposite of the top-down approach and avoids recursion. In this approach, we  solve the problem ``bottom-up" (i.e. by solving all the related sub-problems first).
   \end{alertblock}}     
  \end{itemize}
\end{frame}

\begin{frame}{Algorithms}%this is to create frame
\transblindshorizontal%transistions
\label{algos}%this is label to use in hyperlink
\transdissolve  %transistions

\setbeamercovered{transparent}
\begin{itemize}
    \item Divide and conquer \pause %this is for content present below wont appear 
    \invisible \item Greedy Algorithm \pause%this is for content present below wont appear 
    \invisible \item Dynamic Programming 
\end{itemize}
    
\end{frame}
\begin{frame}%this is to create frame
\transblindshorizontal%transistions
\frametitle{Divide and Conquer}%this is to add title to the frame
\label{dc}%labelling so that we can do cross referencing
Example:\\
\textbf{Quick-Sort:  The average case run time of quick sort is }%this is used to make text bold
$O(n \ast log\ n)$\textbf{.This case happens when we don't exactly
get evenly balanced partitions.}%this is used to make text bold
\end{frame}


\begin{frame}%this is to create frame
\transblindshorizontal%transistions
\frametitle{Divide and Conquer}%this is to add title to the frame
\color{red}{Example:}\\
\color{black}{Merge-Sort: }\color{orange}{The time complexity of Merge Sort is $ O(n \ast log\ n)$. Merge Sort is useful for sorting linked lists in $O(n \ast log\ n)$ time.}%to change textcolor we used \textcolor
\end{frame}


\begin{frame}%this is to create frame
\transblindshorizontal%transistions
\frametitle{Hyperlinks}%this is to add title to the frame
\begin{itemize}%this is to create list of items
\item \hyperlink{algos}{Divide and Conquer}%this is used to make a hyperlink,so when we click on that it will take us to another page
\item  \hyperlink{algos}{\beamergotobutton{Greedy Algorithm}}%this is used to make a hyperlink,so when we click on that it will take us to another page
\item  \hyperlink{algos}{\beamerskipbutton{Dynamic Programming}}%this is used to make a hyperlink,so when we click on that it will take us to another page
\end{itemize}
\end{frame}


\begin{frame}{List of Data Structures}%this is to create frame
\transblindshorizontal%transistions
\setbeamersize{descriptionwidth=0.65cm}
\setbeamercovered{transparent}
    \begin{itemize}%this is to create items
        \item Primitive \pause
        \item Non-Primitive \pause  \newline \\
        \onslide<3->
        \begin{itemize}%this is to create items
            \item \textit{Linear}
            \begin{description}
                \item[$\bullet$]<4-> \textbf{Static}%this is used to make text bold
                \begin{enumerate}%this is to start enumerate
                    \item<5-> Array \newline
                \end{enumerate}
                \item[$\bullet$]<4-> \textbf{Dynamic}%this is used to make text bold
                \begin{enumerate}%this is to start enumerate
                    \item<5-> Linked List
                    \item<5-> Stack
                    \item<5-> Queue \newline
                \end{enumerate}
            \end{description}
            \onslide<6->
            \item \textit{Non-Linear}%this is used to make text italic
            \begin{enumerate}%this is to start enumerate
                \item<6-> Tree
                \item<6-> Graph
            \end{enumerate}
        \end{itemize}
    \end{itemize}
\end{frame}

\begin{frame}{Data Structures}%this is to create frame
\begin{figure}
\centering
\includegraphics[width=9cm]{Image 1.jpeg}%this is to insert a figure 
\caption{1}

\end{figure}
\end{frame}


\begin{frame}%this is to create frame
\transblindshorizontal%transistions
 \begin{table}%this is to create a table

\begin{tabular}{|l|c|c|c|} %used to create a table
\hline 
Algorithm              & Best Case      & Average Case            & Worst Case    \\
\hline
\hline 
Linear Search          &  $O(1)$        & $O(n)$                  &   $O(n)$      \\ 

Binary Search          &  $O(1)$        & $O(log\ n)$             &  $O(log\ n)$  \\ 

Bubble sort            & $O(n)$         &  $O(n^2)$               &$O(n^2)$       \\ 

Selection sort         & $O(n^2)$       &  $O(n^2)$               &$O(n^2)$       \\
 \hline   
\end{tabular}
\caption{1}
\end{table}

\begin{block}{Theorem (Trigonometric Identity)}%this is to create a block
$Sin^2\theta + Cos^2\theta = 1$
\end{block}
\end{frame}


\begin{frame}%this is to create frame
\transblindshorizontal %transistions
\begin{theorem}
    Let a, b, c be lengths of right angled triangle.\newline
    \textbf{By definition}%this is used to make text bold
    \[sin\theta=b/c\left(\frac{oppositeside}{hypotenuse}\right)\]
\[cos\theta=a/c\left(\frac{adjacentside}{hypotenuse}\right)\]
\(sin^2\theta+cos^2\theta=\frac{b^2}{c^2}+\frac{a^2}{c^2}=\frac{a^2+b^2}{c^2}\)\\
    \bigskip%for getting space
    \textbf{From Pythagoras theorem}\newline
 
    $c^2 = a^2 + b^2$ \newline
    
    \(\frac{a^2+b^2}{c^2} = 1\ \Longrightarrow \  sin^2\theta+cos^2\theta = 1\)\\
    \bigskip %for getting space
    \textbf{Hence Proved.}%this is used to make text bold
    \end{theorem}
\end{frame}

\begin{frame}{Multi-line equations}%this is to create frame
\transblindshorizontal%transistions
\begin{align*}%this is used to align equations
    f(x) = x^6 + 7x^3y + 50x^3y^2 & + 12x^2y^4 \\
                                  &- 19x^5y^4 - 10x^7y^6 + 7y^4 - m^3n^3
\end{align*}
\vspace{-5mm}
\begin{align*}%this is used to align equations
%rho,delta,tau are used for signs
\rho \Delta x \Delta y \Delta z \Delta \tau \partial_t c_i(t,x,\tau) 
&= \rho \Delta x \Delta y \Delta z \Delta \tau (p_i-d_i)  \\
&- \rho \Delta y, \Delta z \Delta \tau [q_{i,x}(t,x+\Delta x/2, y, z, \tau)\\
&\qquad - q_{i,x}(t,x - \Delta x/2, y, z, \tau)]\\
&- \rho \Delta x, \Delta z \Delta  \tau [q_{i,y}(t,x,y+\Delta y/2, y, z, \tau)\\
&\qquad - q_{i,y}(t,x,y - \Delta y/2, z, z, \tau)]\\
&- \rho \Delta x \Delta y \Delta \tau[q_{i,z}(t,x,y,z+\Delta z/2, \tau) \\
&\qquad - q_{i,z}(t,x,y,z-\Delta z/2, \tau)]
\end{align*}

\end{frame}

\end{document}